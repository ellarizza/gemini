% Gemini theme
% https://github.com/anishathalye/gemini

\documentclass[final]{beamer}



% ====================
% Packages
% ====================

\usepackage[T1]{fontenc}
\usepackage{lmodern}
\usepackage[size=custom,width=84.1,height=118.9,scale=1.0]{beamerposter}
%\usepackage[size=a0,scale=1.0]{beamerposter}
\usetheme{gemini}
\usecolortheme{gemini}
\usepackage{graphicx}
\usepackage{booktabs}
\usepackage{tikz}
\usepackage{pgfplots}
\pgfplotsset{compat=1.14}
\usepackage{anyfontsize}

% ====================
% Lengths
% ====================

% If you have N columns, choose \sepwidth and \colwidth such that
% (N+1)*\sepwidth + N*\colwidth = \paperwidth
\newlength{\sepwidth}
\newlength{\colwidth}
\setlength{\sepwidth}{0.025\paperwidth}
\setlength{\colwidth}{0.8\paperwidth}

\newcommand{\separatorcolumn}{\begin{column}{\sepwidth}\end{column}}

% ====================
% Title
% ====================

\title{Non-Line of Sight (NLOS) Signal Detection using Recurrent Neural Network for Robust GNSS in Urban Cities}

\author{Ellarizza Fredeluces \inst{1} , Nobuaki Kubo \inst{1}}

\institute[shortinst]{\inst{1} GNSS Laboratory, Tokyo University of Marine Science and Technology}

% ====================
% Footer (optional)
% ====================

\footercontent{
  TUMSAT GNSS Laboratory \hfill
  Marine AI Open Seminar 2025 --- Tokyo University of Marine Science and Technology \hfill
  \href{mailto:d252018@edu.kaiyodai.ac.jp}{d252018@edu.kaiyodai.ac.jp}}
% (can be left out to remove footer)

% ====================
% Logo (optional)
% ====================

% use this to include logos on the left and/or right side of the header:
%\logoright{\includegraphics[height=3cm]{lab_logo.png}}
%\logoleft{\includegraphics[height=3cm]{tumsat_logo2.png}}

% ====================
% Body
% ====================

\begin{document}

\begin{frame}[t]

\begin{block}{BACKGROUND}

  \begin{itemize}
    \item Global Navigation Satellite Systems (GNSS) signals are either blocked or reflected in urban cities like Tokyo.
    \item Range measurements from high altitude satellites can have tens of meters of error. 
    \item These leads to large position error in urban cities.  
    \item \textbf{OBJECTIVE: To improve GNSS position in urban cities by detecting and excluding the satellites with large range errors in computation}
  \end{itemize}
  \begin{figure}
      \centering
      \begin{tikzpicture}[scale=6]
        \draw[step=0.25cm,color=gray] (-1,-1) grid (1,1);
        \draw (1,0) -- (0.2,0.2) -- (0,1) -- (-0.2,0.2) -- (-1,0)
          -- (-0.2,-0.2) -- (0,-1) -- (0.2,-0.2) -- cycle;
      \end{tikzpicture}
      \caption{A figure caption.}
    \end{figure}

\end{block}

\begin{block}{METHODOLOGY}
  \begin{itemize}
    \item \textbf{OBJECTIVE: To improve GNSS position in urban cities by detecting and excluding the satellites with large range errors in computation}
    \item Multiple correlators were generated, labelled, and used to train a recurrent neural network (RNN) model. 
    \item Based on the learned features of a multiple correlator, the model will detect if it's LOS or NLOS. 
    \item When the prediction is NLOS, the measurement will be excluded in position computation. 
  \end{itemize}
  
\end{block}

\begin{block}{RESULTS}
  
\end{block}

\begin{block}{CONCLUSIONS}
  
\end{block}


\end{frame}

\end{document}
